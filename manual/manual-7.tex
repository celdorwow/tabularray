% -*- coding: utf-8 -*-
% !TEX program = lualatex
\documentclass[oneside]{book}

% -*- coding: utf-8 -*-
% !TEX program = lualatex

\newcommand*{\myversion}{2024A}
\newcommand*{\mylpad}[1]{\ifnum#1<10 0\the#1\else\the#1\fi}

\usepackage[a4paper,margin=2.5cm]{geometry}

\setlength{\parindent}{0pt}
\setlength{\parskip}{4pt plus 1pt minus 1pt}

\ExplSyntaxOn
\sys_ensure_backend:
\debug_on:n { check-declarations }
\usepackage{tabularray}
\debug_off:n { check-declarations }
\ExplSyntaxOff

\usepackage{codehigh} % https://ctan.org/pkg/codehigh
\usepackage{array,multirow,amsmath}
\usepackage{chemmacros,environ}
\usepackage{enumitem}

\usepackage[firstpage=true]{background}
\backgroundsetup{contents={}}

\UseTblrLibrary{
  amsmath,booktabs,counter,diagbox,functional,siunitx,varwidth
}

\usepackage{hyperref}
\hypersetup{
  colorlinks=true,
  urlcolor=blue3,
  linkcolor=blue3,
}

\usepackage{tcolorbox}
\tcbset{sharp corners, boxrule=0.5pt, colback=red9}

\usepackage{float}
%\usepackage{enumerate}

\setcounter{tocdepth}{1}

\newcommand*{\K}[1]{\texttt{#1}}
\newcommand*{\V}[1]{\texttt{#1}}
\newcommand*{\None}{$\times$}

\NewTblrEnviron{newtblr}
\SetTblrOuter[newtblr]{long}
\SetTblrInner[newtblr]{
  hlines = {white}, column{1,2} = {co=1}, colsep = 5pt,
  row{odd} = {brown8}, row{even} = {gray8},
  row{1} = {fg=white, bg=purple2, font=\bfseries\sffamily},
}

\NewTblrEnviron{spectblr}
\SetTblrOuter[spectblr]{long}
\SetTblrInner[spectblr]{
  hlines = {white}, column{2} = {co=1}, colsep = 5pt,
  row{odd} = {brown8}, row{even} = {gray8},
  row{1} = {fg=white, bg=purple2, font=\bfseries\sffamily},
  rowhead = 1,
}

\newcommand{\mywarning}[1]{%
  \begin{tcolorbox}
  The interfaces in this #1 should be seen as
  \textcolor{red3}{\bfseries experimental}
  and are likely to change in future releases, if necessary.
  Don’t use them in important documents.
  \end{tcolorbox}
}

%\renewcommand*{\thefootnote}{*}

\colorlet{highback}{azure9}
\CodeHigh{language=latex/table,style/main=highback,style/code=highback}
\NewCodeHighEnv{code}{style/main=gray9,style/code=gray9}
\NewCodeHighEnv{demo}{style/main=gray9,style/code=gray9,demo}

%\CodeHigh{lite}

\CodeHigh{lite}
\setcounter{chapter}{6}

\begin{document}

\chapter{Experimental Interfaces}
\label{chap:exp}

\mywarning{The interfaces in this chapter
(and other undocumented public interfaces even if mentioned in the changelog) should be seen as \textcolor{red3}{\bfseries experimental}
and are likely to change in future releases, if necessary. Don’t use them in important documents.}

\section{Experimental public key paths}

In version 2025A, all \PP{tabularray} key paths were cleaned up as follows:
\begin{itemize}[nosep]
  \item \KP{tabularray/table/inner} (from \KP{tblr})
  \item \KP{tabularray/table/outer} (from \KP{tblr-outer})
  \item \KP{tabularray/column/inner} (from \KP{tblr-column})
  \item \KP{tabularray/row/inner} (from \KP{tblr-row})
  \item \KP{tabularray/cell/inner} (from \KP{tblr-cell-spec})
  \item \KP{tabularray/cell/outer} (from \KP{tblr-cell-span})
  \item \KP{tabularray/hline/inner} (from \KP{tblr-hline})
  \item \KP{tabularray/vline/inner} (from \KP{tblr-vline})
  \item \KP{tabularray/hborder/inner} (from \KP{tblr-hborder})
  \item \KP{tabularray/vborder/inner} (from \KP{tblr-vborder})
\end{itemize}
An advanced user or package writer can use \CC{\DeclareKeys} and \CC{\SetKeys} commands
(provided by LaTeX format) to declare new keys and apply key-value lists, respectively.

The key paths are quite long, therefore \PP{tabularray} provides two shortcut commands
\CC{\DeclareTblrKeys} and \CC{\SetTblrKeys}:

\begin{codehigh}
\DeclareTblrKeys{<path>}{<keyvals>} = \DeclareKeys[tabularray/<path>]{<keyvals>}
\SetTblrKeys{<path>}{<keyvals>} = \SetKeys[tabularray/<path>]{<keyvals>}
\end{codehigh}

\section{Experimental public hook names}

All experimental public \PP{tabularray} hook names provided by \LL{hook} library are as follows:
\begin{itemize}[nosep]
  \item \HP{tabularray/trial/before}
  \item \HP{tabularray/trial/after}
  \item \HP{tabularray/table/before}
  \item \HP{tabularray/table/after}
  \item \HP{tabularray/row/before}
  \item \HP{tabularray/row/after}
  \item \HP{tabularray/cell/before}
  \item \HP{tabularray/cell/after}
\end{itemize}
An advanced user or package writer can use \CC{\AddToHook} and \CC{\AddToHookNext} commands
(provided by LaTeX format) to inject code to \PP{tabularray} tables.

The hook names are quite long, therefore \PP{tabularray} provides two shortcut commands
\CC{\AddToTblrHook} and \CC{\AddToTblrHookNext}:

%\begin{codehigh}
%\AddToTblrHook{<name>}[<label>]{<code>}=\AddToHook{tabularray/<name>}[<label>]{<code>}
%\AddToTblrHookNext{<name>}{<code>}=\AddToHookNext{tabularray/<name>}{<code>}
%\end{codehigh}
\begin{codehigh}
\AddToTblrHook{<name>}{<code>} = \AddToHook{tabularray/<name>}{<code>}
\AddToTblrHookNext{<name>}{<code>} = \AddToHookNext{tabularray/<name>}{<code>}
\end{codehigh}

\section{Experimental public variables}

This variable is always available throughout the whole typesetting process of tables:
\begin{itemize}[nosep]
  \item \CC{\lTblrMeasuringBool}: if \PP{tabularray} is doing trial typesetting.
\end{itemize}
You need to make sure \KV{measure=vstore} to make \CC{\lTblrMeasuringBool} correct.

This variable is available before building every table:
\begin{itemize}[nosep]
  \item \CC{\lTblrPortraitTypeTl}: table type (\VV{short}, \VV{tall} or \VV{long}).
\end{itemize}

These variables are updated in building long tables:
\begin{itemize}[nosep]
  \item \CC{\lTblrRowHeadInt}: total number of head rows.
  \item \CC{\lTblrRowFootInt}: total number of foot rows.
  \item \CC{\lTblrTablePageInt}: index number of current page table.
  \item \CC{\lTblrRowFirstInt}: first row number in row body of current page table.
  \item \CC{\lTblrRowLastInt}: last row number in row body of curent page table.
\end{itemize}

These variables are updated by default before building every cell:
\begin{itemize}[nosep]
  \item \CC{\lTblrCellRowSpanInt}: how many rows are spanned by current cell.
  \item \CC{\lTblrCellColSpanInt}: how many columns are spanned by current cell.
  \item \CC{\lTblrCellOmittedBool}: if current cell is spanned by another cell.
  \item \CC{\lTblrCellBackgroundTl}: background color of current cell.
\end{itemize}

These variables are updated by \LL{html} library before building every cell:
%(you need to write \CC{\UseTblrLibrary{html}} first):
\begin{itemize}[nosep]
  \item \CC{\lTblrCellAboveBorderStyleTl}
  \item \CC{\lTblrCellAboveBorderWidthDim}
  \item \CC{\lTblrCellAboveBorderColorTl}
  \item \CC{\lTblrCellBelowBorderStyleTl}
  \item \CC{\lTblrCellBelowBorderWidthDim}
  \item \CC{\lTblrCellBelowBorderColorTl}
  \item \CC{\lTblrCellLeftBorderStyleTl}
  \item \CC{\lTblrCellLeftBorderWidthDim}
  \item \CC{\lTblrCellLeftBorderColorTl}
  \item \CC{\lTblrCellRightBorderStyleTl}
  \item \CC{\lTblrCellRightBorderWidthDim}
  \item \CC{\lTblrCellRightBorderColorTl}
\end{itemize}
In the above, \TT{BorderStyle}, \TT{BorderWidth}, \TT{BorderColor} are similar to
\K{border-style}, \K{border-width}, \K{border-color} in HTML/CSS, respectively.
\TT{BorderStyle} and \TT{BorderColor} are empty by default.

\section{New child indexers and selectors}

\subsection{One dimensional indexers and selectors}

You can define new child indexers with \CC{\NewTblrChildIndexer} command.
As an example, the following is the simplified definition of \CI{Z} indexer:
\begin{codehigh}
\ExplSyntaxOn
\NewTblrChildIndexer {Z} [1] [1]
  {
    \tl_set:Ne \lTblrChildIndexTl { \int_eval:n {\lTblrChildTotalInt + 1 - #1} {1} }
  }
\ExplSyntaxOff
\end{codehigh}
In the definition, you can use \CC{\lTblrChildTotalInt} which is the total number of children.
And you only need to store the result index \fakeverb{<i>} in \CC{\lTblrChildIndexTl}.
The name of an indexer \underline{MUST} consist of letters and start with an uppercase letter.

You can define new child selectors with \CC{\NewTblrChildSelector} command.
As an example, the following is the simplified definition of \CI{odd} selector:
\begin{codehigh}
\ExplSyntaxOn
\NewTblrChildSelector {odd} [1] [1]
  {
    \clist_set:Ne \lTblrChildClist { {#1} {2} {\int_use:N \lTblrChildTotalInt} }
  }
\ExplSyntaxOff
\end{codehigh}
In the definition, you can use \CC{\lTblrChildTotalInt} which is the total number of children.
And you only need to store the result indexes in \CC{\lTblrChildClist}.
When some indexes form an arithmetic sequence,
you can simplify them as \fakeverb{{<start>}{<step>}{<end>}}.
The name of a selector \underline{MUST} consist of letters and start with a lowercase letter.

\subsection{Two dimensional indexers and selectors}

When selecting cells, you may need two dimensional indexers and selectors.
You can also define new two dimensional child indexers with \CC{\NewTblrChildIndexer} command,
and two dimensional child selectors with \CC{\NewTblrChildSelector} command.

In the definitions, you can use \CC{\lTblrChildHtotalInt}
which is the total number of horizontal children (rows), and \CC{\lTblrChildVtotalInt}
which is the total number of vertical children (columns).

You also need to store the result index \fakeverb{{<i>}{<j>}} in \CC{\lTblrChildIndexTl}
in defining two dimensional child indexers.
Similarly you also need to store the result indexes in \CC{\lTblrChildClist}
in defining two dimensional child selectors.

\subsection{Child ids and classes}

When the table is long, it is clumsy to select children with indexes, positive or negative.
In version 2025A, \PP{tabularray} borrows ideas of ids and classes from HTML/CSS.
With table command \CC{\SetChild}, you can mark a hborder/vborder/row/column/cell with
an id or class, and use it in inner specifications.

The \CC{\SetChild} command accepts a key-value input:
\begin{spectblr}[
  caption = {Key-value input in \fakeverb{\\SetChild} command} %FIXME
]{}
  Input & Description \\
  \KV{id=Hello}  & create a child indexer \TT{Hello} which is an index \TT{{<i>}{<j>}} \\
  \KV{idh=Hello} & create a child indexer \TT{Helloh} which is a horizontal index \TT{<i>} \\
  \KV{idv=Hello} & create a child indexer \TT{Hellov} which is a vertical index  \TT{<j>} \\
  \KV{id*=Hello} & create all of the above three child indexers \\
  \KV{class=world}  & create a child selector \TT{world} which is a list of indexes \TT{{<i>}{<j>}} \\
  \KV{classh=world} & create a child selector \TT{worldh} which is a list of horizontal indexes \TT{<i>} \\
  \KV{classv=world} & create a child selector \TT{worldv} which is a list of vertical indexes \TT{<j>} \\
  \KV{class*=world} & create all of the above three child selectors
\end{spectblr}

The following is an example of child ids
(every id name must start with uppercase letter since it creates a child indexer):

\begin{demohigh}
\begin{tblr}{
  hline{1,Z},
  row{Barh,Quxh} = {bg=azure7},
  column{Bazv,Quxv} = {fg=red3},
  cell{Foo,Qux} = {cmd=\fbox}
}
  1 & 2 & 3 & \SetChild{id=Foo}  4 & 5 & 6 & 7 & 8 \\
  2 & 2 & \SetChild{idh=Bar} 3 & 4 & 5 & 6 & 7 & 8 \\
  3 & \SetChild{idv=Baz} 2 & 3 & 4 & 5 & 6 & 7 & 8 \\
  \SetChild{id*=Qux} 4 & 2 & 3 & 4 & 5 & 6 & 7 & 8
\end{tblr}
\end{demohigh}

The following is an example of child classes
(every class name must start with lowercase letter since it creates a child selector):

\begin{demohigh}
\begin{tblr}{
  hline{1,Z},
  row{fooh} = {bg=azure7},
  column{barv} = {fg=red3},
  cell{baz} = {cmd=\fbox}
}
  1 & 2 & \SetChild{classh=foo} 3 & 4 & 5 & 6 & 7 \\
  2 & \SetChild{classv=bar} 2 & 3 & 4 & 5 & 6 & 7 \\
  \SetChild{class=baz} 3  & 2 & 3 & 4 & 5 & 6 & 7 \\
  4 & 2 & 3 & 4 & \SetChild{class=baz}  5 & 6 & 7 \\
  5 & 2 & 3 & 4 & 5 & \SetChild{classh=foo} 6 & 7 \\
  6 & 2 & 3 & 4 & 5 & 6 & \SetChild{classv=bar} 7
\end{tblr}
\end{demohigh}

Since \CC{\SetChild} commands need to be extracted first
before parsing inner specifications, they \emph{must} be put
at the beginning of cells, before other table commands such as \CC{\hline}.
Therefore it conflicts with syntaxes \fakeverb{\\\\[<dimen>]} and \fakeverb{\\\\*}.
They can be replaced with \fakeverb{\\\\\SetRow{belowsep+=<dimen>}}
and \fakeverb{\\\\\nopagebreak} respectively,
so that \CC{\SetChild} can be inserted in the middle:
\begin{demohigh}
\begin{tblr}{
  cell{Foo,Bar,Baz,Qux} = {cmd=\fbox}
}
\SetChild{id=Foo}\hline
  1 & 2 & 3 & 4 & 5 & 6 & 7 \\
\SetChild{id=Bar}\nopagebreak\hline
  2 & 2 & 3 & 4 & 5 & 6 & 7 \\
\SetChild{id=Baz}\SetRow{belowsep+=5pt}\hline
  3 & 2 & 3 & 4 & 5 & 6 & 7 \\
\SetChild{id=Qux}\hline
\end{tblr}
\end{demohigh}

Only one \CC{\SetChild} command in each cell is supported.
But you can create multiple ids or classes with single \CC{\SetChild} command.

In drawing \PP{tikz} pictures on tables (see Chapter~\ref{chap:lib}),
you may want to get the value of a child id or class with \CC{\ExpTblrChildId} or \CC{\ExpTblrChildClass}.
These two commands are fully expandable.

\end{document}
