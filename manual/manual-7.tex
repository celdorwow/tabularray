% -*- coding: utf-8 -*-
% !TEX program = lualatex
\documentclass[oneside]{book}

% -*- coding: utf-8 -*-
% !TEX program = lualatex

\newcommand*{\myversion}{2024A}
\newcommand*{\mylpad}[1]{\ifnum#1<10 0\the#1\else\the#1\fi}

\usepackage[a4paper,margin=2.5cm]{geometry}

\setlength{\parindent}{0pt}
\setlength{\parskip}{4pt plus 1pt minus 1pt}

\ExplSyntaxOn
\sys_ensure_backend:
\debug_on:n { check-declarations }
\usepackage{tabularray}
\debug_off:n { check-declarations }
\ExplSyntaxOff

\usepackage{codehigh} % https://ctan.org/pkg/codehigh
\usepackage{array,multirow,amsmath}
\usepackage{chemmacros,environ}
\usepackage{enumitem}

\usepackage[firstpage=true]{background}
\backgroundsetup{contents={}}

\UseTblrLibrary{
  amsmath,booktabs,counter,diagbox,functional,siunitx,varwidth
}

\usepackage{hyperref}
\hypersetup{
  colorlinks=true,
  urlcolor=blue3,
  linkcolor=blue3,
}

\usepackage{tcolorbox}
\tcbset{sharp corners, boxrule=0.5pt, colback=red9}

\usepackage{float}
%\usepackage{enumerate}

\setcounter{tocdepth}{1}

\newcommand*{\K}[1]{\texttt{#1}}
\newcommand*{\V}[1]{\texttt{#1}}
\newcommand*{\None}{$\times$}

\NewTblrEnviron{newtblr}
\SetTblrOuter[newtblr]{long}
\SetTblrInner[newtblr]{
  hlines = {white}, column{1,2} = {co=1}, colsep = 5pt,
  row{odd} = {brown8}, row{even} = {gray8},
  row{1} = {fg=white, bg=purple2, font=\bfseries\sffamily},
}

\NewTblrEnviron{spectblr}
\SetTblrOuter[spectblr]{long}
\SetTblrInner[spectblr]{
  hlines = {white}, column{2} = {co=1}, colsep = 5pt,
  row{odd} = {brown8}, row{even} = {gray8},
  row{1} = {fg=white, bg=purple2, font=\bfseries\sffamily},
  rowhead = 1,
}

\newcommand{\mywarning}[1]{%
  \begin{tcolorbox}
  The interfaces in this #1 should be seen as
  \textcolor{red3}{\bfseries experimental}
  and are likely to change in future releases, if necessary.
  Don’t use them in important documents.
  \end{tcolorbox}
}

%\renewcommand*{\thefootnote}{*}

\colorlet{highback}{azure9}
\CodeHigh{language=latex/table,style/main=highback,style/code=highback}
\NewCodeHighEnv{code}{style/main=gray9,style/code=gray9}
\NewCodeHighEnv{demo}{style/main=gray9,style/code=gray9,demo}

%\CodeHigh{lite}

\CodeHigh{lite}
\setcounter{chapter}{6}

\begin{document}

\chapter{Experimental Interfaces}
\label{chap:exp}

\mywarning{chapter}

\section{Experimental Public Key Paths}

In version 2025A, all \PP{tabularray} key paths were cleaned up as follows:
\begin{itemize}[nosep]
  \item \KP{tabularray/table/inner} (from \KP{tblr})
  \item \KP{tabularray/table/outer} (from \KP{tblr-outer})
  \item \KP{tabularray/column/inner} (from \KP{tblr-column})
  \item \KP{tabularray/row/inner} (from \KP{tblr-row})
  \item \KP{tabularray/cell/inner} (from \KP{tblr-cell-spec})
  \item \KP{tabularray/cell/outer} (from \KP{tblr-cell-span})
  \item \KP{tabularray/hline/inner} (from \KP{tblr-hline})
  \item \KP{tabularray/vline/inner} (from \KP{tblr-vline})
  \item \KP{tabularray/hborder/inner} (from \KP{tblr-hborder})
  \item \KP{tabularray/vborder/inner} (from \KP{tblr-vborder})
\end{itemize}
An advanced user or package writer can use \CC{\DeclareKeys} and \CC{\SetKeys} commands
(provided by LaTeX format) to declare new keys and apply key-value lists, respectively.

\section{Experimental Public Hook Paths}

All experimental public \PP{tabularray} hooks provided by \LL{hook} library are as follows:
\begin{itemize}[nosep]
  \item \HP{tabularray/trial/before}
  \item \HP{tabularray/trial/after}
  \item \HP{tabularray/table/before}
  \item \HP{tabularray/table/after}
  \item \HP{tabularray/row/before}
  \item \HP{tabularray/row/after}
  \item \HP{tabularray/cell/before}
  \item \HP{tabularray/cell/after}
\end{itemize}
An advanced user or package writer can use \CC{\AddToHook} and \CC{\AddToHookNext} commands
(provided by LaTeX format) to inject code to \PP{tabularray} tables.

\section{Experimental Public Variables}

This variable is always available throughout the whole typesetting process of tables:
\begin{itemize}[nosep]
  \item \CC{\lTblrMeasuringBool}: if \PP{tabularray} is doing trial typesetting.
\end{itemize}
You need to make sure \KV{measure=vstore} to make \CC{\lTblrMeasuringBool} correct.

These variables are updated by default before building every cell:
\begin{itemize}[nosep]
  \item \CC{\lTblrCellRowSpanTl}: how many rows are spanned by current cell.
  \item \CC{\lTblrCellColSpanTl}: how many columns are spanned by current cell.
  \item \CC{\lTblrCellOmittedBool}: if current cell is spanned by another cell.
  \item \CC{\lTblrCellBackgroundTl}: background color of current cell.
\end{itemize}

These variables are updated by \LL{html} library before building every cell:
%(you need to write \CC{\UseTblrLibrary{html}} first):
\begin{itemize}[nosep]
  \item \CC{\lTblrCellAboveBorderStyleTl}
  \item \CC{\lTblrCellAboveBorderWidthTl}
  \item \CC{\lTblrCellAboveBorderColorTl}
  \item \CC{\lTblrCellBelowBorderStyleTl}
  \item \CC{\lTblrCellBelowBorderWidthTl}
  \item \CC{\lTblrCellBelowBorderColorTl}
  \item \CC{\lTblrCellLeftBorderStyleTl}
  \item \CC{\lTblrCellLeftBorderWidthTl}
  \item \CC{\lTblrCellLeftBorderColorTl}
  \item \CC{\lTblrCellRightBorderStyleTl}
  \item \CC{\lTblrCellRightBorderWidthTl}
  \item \CC{\lTblrCellRightBorderColorTl}
\end{itemize}
In the above, \TT{BorderStyle}, \TT{BorderWidth}, \TT{BorderColor} are similar to
\K{border-style}, \K{border-width}, \K{border-color} in HTML/CSS, respectively.
\TT{BorderStyle} and \TT{BorderColor} are empty by default.

\end{document}
