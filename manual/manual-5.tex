% -*- coding: utf-8 -*-
% !TEX program = lualatex
\documentclass[oneside]{book}

% -*- coding: utf-8 -*-
% !TEX program = lualatex

\newcommand*{\myversion}{2024A}
\newcommand*{\mylpad}[1]{\ifnum#1<10 0\the#1\else\the#1\fi}

\usepackage[a4paper,margin=2.5cm]{geometry}

\setlength{\parindent}{0pt}
\setlength{\parskip}{4pt plus 1pt minus 1pt}

\ExplSyntaxOn
\sys_ensure_backend:
\debug_on:n { check-declarations }
\usepackage{tabularray}
\debug_off:n { check-declarations }
\ExplSyntaxOff

\usepackage{codehigh} % https://ctan.org/pkg/codehigh
\usepackage{array,multirow,amsmath}
\usepackage{chemmacros,environ}
\usepackage{enumitem}

\usepackage[firstpage=true]{background}
\backgroundsetup{contents={}}

\UseTblrLibrary{
  amsmath,booktabs,counter,diagbox,functional,siunitx,varwidth
}

\usepackage{hyperref}
\hypersetup{
  colorlinks=true,
  urlcolor=blue3,
  linkcolor=blue3,
}

\usepackage{tcolorbox}
\tcbset{sharp corners, boxrule=0.5pt, colback=red9}

\usepackage{float}
%\usepackage{enumerate}

\setcounter{tocdepth}{1}

\newcommand*{\K}[1]{\texttt{#1}}
\newcommand*{\V}[1]{\texttt{#1}}
\newcommand*{\None}{$\times$}

\NewTblrEnviron{newtblr}
\SetTblrOuter[newtblr]{long}
\SetTblrInner[newtblr]{
  hlines = {white}, column{1,2} = {co=1}, colsep = 5pt,
  row{odd} = {brown8}, row{even} = {gray8},
  row{1} = {fg=white, bg=purple2, font=\bfseries\sffamily},
}

\NewTblrEnviron{spectblr}
\SetTblrOuter[spectblr]{long}
\SetTblrInner[spectblr]{
  hlines = {white}, column{2} = {co=1}, colsep = 5pt,
  row{odd} = {brown8}, row{even} = {gray8},
  row{1} = {fg=white, bg=purple2, font=\bfseries\sffamily},
  rowhead = 1,
}

\newcommand{\mywarning}[1]{%
  \begin{tcolorbox}
  The interfaces in this #1 should be seen as
  \textcolor{red3}{\bfseries experimental}
  and are likely to change in future releases, if necessary.
  Don’t use them in important documents.
  \end{tcolorbox}
}

%\renewcommand*{\thefootnote}{*}

\colorlet{highback}{azure9}
\CodeHigh{language=latex/table,style/main=highback,style/code=highback}
\NewCodeHighEnv{code}{style/main=gray9,style/code=gray9}
\NewCodeHighEnv{demo}{style/main=gray9,style/code=gray9,demo}

%\CodeHigh{lite}

\CodeHigh{lite}
\setcounter{chapter}{4}

\begin{document}

\chapter{Use Some Libraries}

A \PP{tabularray} library could be loaded by \CC{\UseTblrLibrary} command.
From version 2025A, an external library \fakeverb{foo} could also be loaded
if its filename is \FF{tblrlibfoo.sty}.

\section{Library \texttt{amsmath}}

With \CC{\UseTblrLibrary{amsmath}} in the preamble of the document,
\PP{tabularray} will load \PP{amsmath} package, and define \EE{+array}, \EE{+matrix},
\EE{+bmatrix}, \EE{+Bmatrix}, \EE{+pmatrix}, \EE{+vmatrix}, \EE{+Vmatrix} and \EE{+cases}
environments. Each of the environments is similar to the environment without \TT{+} prefix in its name,
but has default \KV{rowsep=2pt} just as \EE{tblr} environment. Every environment
except \EE{+array} accepts an optional argument, where you can write inner specifications.

\begin{demo}
$\begin{pmatrix}
 \dfrac{2}{3} &  \dfrac{2}{3} &  \dfrac{1}{3} \\
 \dfrac{2}{3} & -\dfrac{1}{3} & -\dfrac{2}{3} \\
 \dfrac{1}{3} & -\dfrac{2}{3} &  \dfrac{2}{3} \\
\end{pmatrix}$
\end{demo}

\begin{demohigh}
$\begin{+pmatrix}[cells={r},row{2}={purple8}]
 \dfrac{2}{3} &  \dfrac{2}{3} &  \dfrac{1}{3} \\
 \dfrac{2}{3} & -\dfrac{1}{3} & -\dfrac{2}{3} \\
 \dfrac{1}{3} & -\dfrac{2}{3} &  \dfrac{2}{3} \\
\end{+pmatrix}$
\end{demohigh}

\begin{demo}
$f(x)=\begin{cases}
 0,            & x=1; \\
 \dfrac{1}{3}, & x=2; \\
 \dfrac{2}{3}, & x=3; \\
 1,            & x=4. \\
\end{cases}$
\end{demo}

\begin{demohigh}
$f(x)=\begin{+cases}
 0,            & x=1; \\
 \dfrac{1}{3}, & x=2; \\
 \dfrac{2}{3}, & x=3; \\
 1,            & x=4. \\
\end{+cases}$
\end{demohigh}

\section{Library \texttt{booktabs}}

With \CC{\UseTblrLibrary{booktabs}} in the preamble of the document,
\PP{tabularray} will load \PP{booktabs} package,
and define
  \CC{\toprule}, \CC{\midrule}, \CC{\bottomrule},
  \CC{\cmidrule}, \CC{\cmidrulemore}, \CC{\morecmidrules},
  \CC{\specialrule}, \CC{\addrowspace}, and \CC{\addlinespace}
as table commands.

\begin{demohigh}
\begin{tblr}{llll}
\toprule
 Alpha   & Beta  & Gamma   & Delta \\
\midrule
 Epsilon & Zeta  & Eta     & Theta \\
\cmidrule{1-3}
 Iota    & Kappa & Lambda  & Mu    \\
\cmidrule{2-4}
 Nu      & Xi    & Omicron & Pi    \\
\bottomrule
\end{tblr}
\end{demohigh}

Just like \CC{\hline} and \CC{\cline} commands,
you can also specify rule width and color by using hline keys in the optional
argument of any of these commands.

Like in \PP{booktabs}, by default
  width of \CC{\toprule} and \CC{\bottomrule} are determined by \CC{\heavyrulewidth},
  width of \CC{\midrule} is determined by \CC{\lightrulewidth}, and
  width of \CC{\cmidrule} and \CC{\cmidrulemore} are determined by \CC{\cmidrulewidth}, respectively.
All three \CC{\...rulewidth} are dimensions.

\begin{demohigh}
\begin{tblr}{llll}
\toprule[2pt,purple3]
 Alpha   & Beta  & Gamma  & Delta \\
\midrule[blue3]
 Epsilon & Zeta  & Eta    & Theta \\
\cmidrule[azure3]{2-3}
 Iota    & Kappa & Lambda & Mu    \\
\bottomrule[2pt,purple3]
\end{tblr}
\end{demohigh}

If you need more than one \verb!\cmidrule!s, you can use \verb!\cmidrulemore!
command, which is simpler than the \verb!booktabs! usage
\verb!\morecmidrules\cmidrule!.
\verb!\cmidrulemore! can receive hline keys in an optional argument too.

\begin{demohigh}
\begin{tblr}{llll}
\toprule
 Alpha   & Beta  & Gamma   & Delta \\
\cmidrule{1-3} \cmidrulemore{2-4}
 Epsilon & Zeta  & Eta     & Theta \\
\cmidrule{1-3} \morecmidrules \cmidrule{2-4}
 Iota    & Kappa & Lambda  & Mu    \\
\bottomrule
\end{tblr}
\end{demohigh}

From version 2021N (2021-09-01), you can set trimming positions of
\CC{\cmidrule} and \CC{\cmidrulemore}, using newly introduced trimming
options (\K{leftpos}, \K{rightpos}, \K{endpos}, \K{l}, \K{r},
and \K{lr}) (see Section~\ref{sec:hlines-vlines}).
Option \K{endpos} is already applied to these two commands.

\begin{demohigh}
\begin{tblr}{llll}
\toprule
 Alpha   & Beta  & Gamma   & Delta \\
\cmidrule[lr]{1-2} \cmidrule[lr=-0.4]{3-4}
 Epsilon & Zeta  & Eta     & Theta \\
\cmidrule[r]{1-2} \cmidrule[l]{3-4}
 Iota    & Kappa & Lambda  & Mu    \\
\bottomrule
\end{tblr}
\end{demohigh}

Since \PP{booktabs} tables usually don't have vlines, the meaningful values
here are decimal numbers between \V{-1} and \V{0}.
The default value \V{-0.8} for \V{l}, \V{r}, and \V{lr} is chosen to
make similar result as \PP{booktabs} package does.

There is also a \EE{booktabs} environment for you. With this environment,
the default \KV{rowsep=0pt}, but extra vertical space will be added by
\CC{\toprule}, \CC{\midrule}, \CC{\bottomrule} and \CC{\cmidrule} commands.
The sizes of vertical space are determined by \CC{\aboverulesep} and \CC{\belowrulesep} dimensions.

\begin{demohigh}
\begin{booktabs}{
  colspec = lcccc,
  cell{1}{1} = {r=2}{}, cell{1}{2,4} = {c=2}{},
}
\toprule
  Sample & I &   & II &   \\
\cmidrule[lr]{2-3} \cmidrule[lr]{4-5}
         & A & B & C & D \\
\midrule
  S1     & 5 & 6 & 7 & 8 \\
  S2     & 6 & 7 & 8 & 5 \\
  S3     & 7 & 8 & 5 & 6 \\
\bottomrule
\end{booktabs}
\end{demohigh}
% S4     & 8 & 5 & 6 & 7 \\

You can also use \CC{\specialrule} command.
The second argument sets \K{belowsep} of previous row,
and the third argument sets \K{abovesep} of current row,

\begin{demohigh}
\begin{booktabs}{row{2}={olive9}}
\toprule
 Alpha   & Beta  & Gamma   & Delta \\
\specialrule{0.5pt}{4pt}{6pt}
 Epsilon & Zeta  & Eta     & Theta \\
\specialrule{0.8pt,blue3}{3pt}{2pt}
 Iota    & Kappa & Lambda  & Mu    \\
\bottomrule
\end{booktabs}
\end{demohigh}

At last, there is also an \CC{\addlinespace} command, with an alternative
name \CC{\addrowspace}.
You can specify the size of vertical space to be added in its optional
argument, and the default size is determinted by \CC{\defaultaddspace}
dimension, initially \V{0.5em}.
This command adds one half of the space to \K{belowsep} of previous row,
and the other half to \K{abovesep} of current row.

\begin{demohigh}
\begin{booktabs}{row{2}={olive9}}
\toprule
 Alpha   & Beta  & Gamma   & Delta \\
\addlinespace
 Epsilon & Zeta  & Eta     & Theta \\
\addlinespace[1em]
 Iota    & Kappa & Lambda  & Mu    \\
\bottomrule
\end{booktabs}
\end{demohigh}

From version 2022A (2022-03-01), there is a \EE{longtabs} environment
for writing long \PP{booktabs} tables,
and a \EE{talltabs} environment for writing tall \PP{booktabs} tables.

\section{Library \texttt{counter}}

You need to load \LL{counter} library with \CC{\UseTblrLibrary{counter}},
if you want to modify some LaTeX counters inside \PP{tabularray} tables.

\begin{demohigh}
\newcounter{mycnta}
\newcommand{\mycnta}{\stepcounter{mycnta}\arabic{mycnta}}
\begin{tblr}{hlines}
  \mycnta & \mycnta & \mycnta \\
  \mycnta & \mycnta & \mycnta \\
  \mycnta & \mycnta & \mycnta \\
\end{tblr}
\end{demohigh}

\section{Library \texttt{diagbox}}

When writing \CC{\UseTblrLibrary{diagbox}} in the preamble of the document,
\PP{tabularray} package loads \PP{diagbox} package,
and you can use \CC{\diagbox} and \CC{\diagboxthree} commands inside \EE{tblr} environment.

\begin{demohigh}
\begin{tblr}{hlines,vlines}
 \diagbox{Aa}{Pp} & Beta & Gamma \\
 Epsilon & Zeta  & Eta \\
 Iota    & Kappa & Lambda \\
\end{tblr}
\end{demohigh}

\begin{demohigh}
\begin{tblr}{hlines,vlines}
 \diagboxthree{Aa}{Pp}{Hh} & Beta & Gamma \\
 Epsilon & Zeta  & Eta \\
 Iota    & Kappa & Lambda \\
\end{tblr}
\end{demohigh}

You can also use \CC{\diagbox} and \CC{\diagboxthree} commands in math mode.
\nopagebreak
\begin{demohigh}
$\begin{tblr}{|c|cc|}
\hline
 \diagbox{X_1}{X_2} & 0 & 1 \\
\hline
  0 & 0.1 & 0.2 \\
  1 & 0.3 & 0.4 \\
\hline
\end{tblr}$
\end{demohigh}

\section{Library \texttt{functional}}

With \CC{\UseTblrLibrary{functional}} in the preamble of the document,
\PP{tabularray} will load \href{https://ctan.org/pkg/functional}{\PP{functional}} package,
and define outer key \K{evaluate} and inner key \K{process}.
This library brings intuitive functional programming into \PP{tabularray} tables.

\subsection{Evaluate inner specifications}

With this library, \PP{tabularray} will evaluate every function
(defined with \CC{\prgNewFunction}) within inner specifications,
replacing it with its return value, before parsing the key-value pairs. Here is an example:
\begin{demohigh}
\begin{tblr}{
  hlines,
  row{2} = {bg=\funColor{RGB}{180,180,255}}
}
  Alpha   & Beta  & Gamma  \\
  Epsilon & Zeta  & Eta    \\
  Iota    & Kappa & Lambda
\end{tblr}
\end{demohigh}
The \CC{\funColor} function is provided by \PP{functional} package.
And now let's see another example:
\begin{demohigh}
\begin{tblr}{
  row{2} = {bg=\intIfOddTF{\value{page}}{\prgReturn{red7}}{\prgReturn{blue7}}}
}
  Alpha   & Beta  & Gamma  \\
  Epsilon & Zeta  & Eta    \\
  Iota    & Kappa & Lambda \\
\end{tblr}
\end{demohigh}

You may like to define a new function for it if you need to use it several times:

\begin{demohigh}
\IgnoreSpacesOn
\prgNewFunction \colorMagic {mm} {
  \intIfOddTF{\value{page}}{\prgReturn{#1}}{\prgReturn{#2}}
}
\IgnoreSpacesOff
\begin{tblr}{
  row{1} = {bg=\colorMagic{yellow7}{brown7}},
  row{3} = {bg=\colorMagic{green7}{teal7}}
}
  Alpha   & Beta  & Gamma  \\
  Epsilon & Zeta  & Eta    \\
  Iota    & Kappa & Lambda \\
\end{tblr}
\end{demohigh}

\subsection{Evaluate table body}

With outer key \K{evaluate}, you can evaluate every occurrence of a specified protected function
(defined with \CC{\prgNewFunction}) and replace it with the return value before splitting the table body.

The first application of \K{evaluate} key is for inputting files inside tables.
Assume you have two files \FF{test1.tmp} and \FF{test2.tmp} with the following contents:

\begin{filecontents*}[overwrite]{test1.tmp}
Some & Some \\
\end{filecontents*}
\begin{filecontents*}[overwrite]{test2.tmp}
Other & Other \\
\end{filecontents*}

\begin{codehigh}
\begin{filecontents*}[overwrite]{test1.tmp}
Some & Some \\
\end{filecontents*}
\end{codehigh}

\begin{codehigh}
\begin{filecontents*}[overwrite]{test2.tmp}
Other & Other \\
\end{filecontents*}
\end{codehigh}

Then you can input them with outer specification \KV{evaluate=\fileInput}.
The \CC{\fileInput} function is provided by \PP{functional} package.

\begin{demohigh}
\begin{tblr}[evaluate=\fileInput]{hlines}
  Row1 & 1 \\
  \fileInput{test1.tmp}
  Row3 & 3 \\
  \fileInput{test2.tmp}
  Row5 & 5 \\
\end{tblr}
\end{demohigh}

In general, you can define your functions which return parts of table contents,
and use \K{evaluate} key to evaluate them inside tables.

\begin{demohigh}
\IgnoreSpacesOn
\prgNewFunction \myFunOne {m} {
  \prgReturn {#1 & #1 \\}
}
\IgnoreSpacesOff
\begin{tblr}[evaluate=\myFunOne]{hlines}
  Row1 & 1 \\
  \myFunOne{Text}
  Row3 & 3 \\
  \myFunOne{Text}
  Row5 & 5 \\
\end{tblr}
\end{demohigh}

\begin{demohigh}
\IgnoreSpacesOn
\prgNewFunction \myFunTwo {} {
  \prgReturn {Other & Other \\}
}
\IgnoreSpacesOff
\begin{tblr}[evaluate=\myFunTwo]{hlines}
  Row1 & 1 \\
  \myFunTwo
  Row3 & 3 \\
  \myFunTwo
  Row5 & 5 \\
\end{tblr}
\end{demohigh}

You can even generate the whole table with some function.

\begin{demohigh}
\IgnoreSpacesOn
\prgNewFunction \makeEmptyTable {mm}  {
  \tlSet \lTmpaTl {\intReplicate {\intEval{#2-1}} {&}}
  \tlPutRight \lTmpaTl {\\}
  \intReplicate {#1} {\tlUse \lTmpaTl}
}
\IgnoreSpacesOff
\begin{tblr}[evaluate=\makeEmptyTable]{hlines,vlines}
  \makeEmptyTable{3}{7}
\end{tblr}
\end{demohigh}

From version 2023A, you can evaluate all functions in the table body
with option \texttt{evaluate=all}.

\subsection{Process table elements}

With inner key \K{process}, you can modify the contents and styles before the table is built.
Several public functions defined with \CC{\prgNewFuncton} are provided for you:

\begin{itemize}
  \item \CC{\cellGetText{<rownum>}{<colnum>}}
  \item \CC{\cellSetText{<rownum>}{<colnum>}{<text>}}
  \item \CC{\cellSetStyle{<rownum>}{<colnum>}{<style>}}
  \item \CC{\rowSetStyle{<rownum>}{<style>}}
  \item \CC{\columnSetStyle{<colnum>}{<style>}}
\end{itemize}

As the first example, let's calculate the sums of cells column by column:

\IgnoreSpacesOn
\prgNewFunction \calcSum {} {
  \intStepOneInline {1} {\arabic{colcount}} {
    \intZero \lTmpaInt
    \intStepOneInline {1} {\arabic{rowcount}-1} {
      \intAdd \lTmpaInt {\cellGetText {####1} {##1}}
    }
    \cellSetText {\expWhole{\arabic{rowcount}}} {##1} {\intUse\lTmpaInt}
  }
}
\IgnoreSpacesOff
\begin{codehigh}
\IgnoreSpacesOn
\prgNewFunction \calcSum {} {
  \intStepOneInline {1} {\arabic{colcount}} {
    \intZero \lTmpaInt
    \intStepOneInline {1} {\arabic{rowcount}-1} {
      \intAdd \lTmpaInt {\cellGetText {####1} {##1}}
    }
    \cellSetText {\expWhole{\arabic{rowcount}}} {##1} {\intUse\lTmpaInt}
  }
}
\IgnoreSpacesOff
\end{codehigh}

\begin{demohigh}
\begin{tblr}{colspec={rrr},process=\calcSum}
\hline
  1 & 2 & 3 \\
  4 & 5 & 6 \\
  7 & 8 & 9 \\
\hline
    &   &   \\
\hline
\end{tblr}
\end{demohigh}

Now, let's set background colors of cells depending on their contents:

\IgnoreSpacesOn
\prgNewFunction \colorBack {} {
  \intStepOneInline {1} {\arabic{rowcount}} {
    \intStepOneInline {1} {\arabic{colcount}} {
      \intSet \lTmpaInt {\cellGetText {##1} {####1}}
      \intCompareTF {\lTmpaInt} > {0}
        {\cellSetStyle {##1} {####1} {bg=purple8}}
        {\cellSetStyle {##1} {####1} {bg=olive8}}
    }
  }
}
\IgnoreSpacesOff
\begin{codehigh}
\IgnoreSpacesOn
\prgNewFunction \colorBack {} {
  \intStepOneInline {1} {\arabic{rowcount}} {
    \intStepOneInline {1} {\arabic{colcount}} {
      \intSet \lTmpaInt {\cellGetText {##1} {####1}}
      \intCompareTF {\lTmpaInt} > {0}
        {\cellSetStyle {##1} {####1} {bg=purple8}}
        {\cellSetStyle {##1} {####1} {bg=olive8}}
    }
  }
}
\IgnoreSpacesOff
\end{codehigh}

\begin{demohigh}
\begin{tblr}{hlines,vlines,cells={r,$},process=\colorBack}
  -1 &  2 &  3 \\
   4 &  5 & -6 \\
   7 & -8 &  9 \\
\end{tblr}
\end{demohigh}

We can also use color series of \PP{xcolor} package to color table rows:

\definecolor{lightb}{RGB}{217,224,250}
\definecolorseries{tblrow}{rgb}{last}{lightb}{white}
\resetcolorseries[3]{tblrow}
\IgnoreSpacesOn
\prgNewFunction \colorSeries {} {
  \intStepOneInline {1} {\arabic{rowcount}} {
    \tlSet \lTmpaTl {\intMathMod {##1-1} {3}}
    \rowSetStyle {##1} {\expWhole{bg=tblrow!![\lTmpaTl]}}
  }
}
\IgnoreSpacesOff
\begin{codehigh}
\definecolor{lightb}{RGB}{217,224,250}
\definecolorseries{tblrow}{rgb}{last}{lightb}{white}
\resetcolorseries[3]{tblrow}
\IgnoreSpacesOn
\prgNewFunction \colorSeries {} {
  \intStepOneInline {1} {\arabic{rowcount}} {
    \tlSet \lTmpaTl {\intMathMod {##1-1} {3}}
    \rowSetStyle {##1} {\expWhole{bg=tblrow!![\lTmpaTl]}}
  }
}
\IgnoreSpacesOff
\end{codehigh}

\begin{demohigh}
\begin{tblr}{hlines,process=\colorSeries}
  Row1 & 1 \\
  Row2 & 2 \\
  Row3 & 3 \\
  Row4 & 4 \\
  Row5 & 5 \\
  Row6 & 6 \\
\end{tblr}
\end{demohigh}

\section{Library \texttt{hook}}

This library is \textcolor{red3}{experimental}.
It will also load \LL{varwidth} library and set \KV{measure=vstore} as default.
See Chapter \ref{chap:exp} for more details of the library.

\section{Library \texttt{html}}

This library is \textcolor{red3}{experimental}.
See Chapter \ref{chap:exp} for more details of the library.

\section{Library \texttt{nameref}}

From version 2022D, you can load \LL{nameref} library
to make \CC{\nameref} and \EE{longtblr} work together.

\section{Library \texttt{siunitx}}

When writing \CC{\UseTblrLibrary{siunitx}} in the preamble of the document,
\PP{tabularray} package loads \PP{siunitx} package,
and defines \Q{S} column as \Q{Q} column with \K{si} key.

\begin{demohigh}
\begin{tblr}{
  hlines, vlines,
  colspec={S[table-format=3.2]S[table-format=3.2]}
}
   {Head}  & {Head} \\
  111      & 111    \\
    2.1    &   2.2  \\
   33.11   &  33.22 \\
\end{tblr}
\end{demohigh}

\begin{demohigh}
\begin{tblr}{
  hlines, vlines,
  colspec={Q[si={table-format=3.2},c]Q[si={table-format=3.2},c]}
}
   {Head}  & {Head} \\
  111      & 111    \\
    2.1    &   2.2  \\
   33.11   &  33.22 \\
\end{tblr}
\end{demohigh}

Note that you need to use one pairs of curly braces to guard non-numeric cells%
\footnote{Before version 2025A, three pairs of braces are needed.}.
But it is cumbersome to enclose each cell with braces. From version 2022B (2022-06-01)
a new key \K{guard} is provided for cells and rows. With \K{guard} key the previous example
can be largely simplified.

\begin{demohigh}
\begin{tblr}{
  hlines, vlines,
  colspec={Q[si={table-format=3.2},c]Q[si={table-format=3.2},c]},
  row{1} = {guard}
}
   Head  & Head   \\
  111    & 111    \\
    2.1  &   2.2  \\
   33.11 &  33.22 \\
\end{tblr}
\end{demohigh}

Also you must use \K{l}, \K{c} or \K{r} to set horizontal alignment for non-numeric cells:
\nopagebreak
\begin{demohigh}
\begin{tblr}{
  hlines, vlines, columns={6em},
  colspec={
    Q[si={table-format=3.2,table-number-alignment=left},l,blue7]
    Q[si={table-format=3.2,table-number-alignment=center},c,teal7]
    Q[si={table-format=3.2,table-number-alignment=right},r,purple7]
  },
  row{1} = {guard}
}
  Head  & Head   & Head   \\
 111    & 111    & 111    \\
   2.1  &   2.2  &   2.3  \\
  33.11 &  33.22 &  33.33 \\
\end{tblr}
\end{demohigh}

Both \Q{S} and \Q{s} columns are supported. In fact, These two columns have been defined as follows:
\begin{codehigh}
\NewColumnType{S}[1][]{Q[si={#1},c]}
\NewColumnType{s}[1][]{Q[si={#1},c,cmd=\TblrUnit]}
\end{codehigh}
You don't need to and are not allowed to define them again.

\section{Library \texttt{tikz}}

With this \textcolor{red3}{experimental} \LL{tikz} library%
\footnote{The author thanks \href{https://github.com/jasperhabicht}{Jasper Habicht}
for his contributions to this library.},
you can draw \PP{tikz} pictures below or above \PP{tabularray} (short or tall) tables.
(Support for long tables is rather complicated hence not available at this time.)
Note that this library depends on and loads \PP{hook} library.

To draw below/above a \PP{tabularray} table,
write some \PP{tikz} code inside \EE{tblrtikzbelow}/\EE{tblrtikzabove} environment.
Both of them should be put before the table, and two compilations are needed to get desired result.

Inside \EE{tblrtikzbelow}/\EE{tblrtikzabove} environment, you can use these predefined nodes:

\begin{spectblr}[
  caption = {Nodes created by \PP{tikz} library}
]{}
  Node Name &  Node Description \\
  table node \NN{table}  & rectangle node for the whole table \\
  cell node \NN{<i>-<j>} & rectangle node for \KK{cell{<i>}{<j>}} \\
  corner node \NN{h<i>}  & coordinate node at the instersection point
                           of \KK{hborder{<i>}} and \KK{vborder{1}} \\
  corner node \NN{v<j>}  & coordinate node at the instersection point
                           of \KK{vborder{<j>}} and \KK{hborder{1}}
\end{spectblr}

The first example below demonstrates the table node and cell nodes:

\begin{demohigh}
\begin{tblrtikzbelow}
  \path[pattern color=lightgray,pattern=checkerboard,
        draw=blue3, ultra thick, rounded corners]
    (table.north west) rectangle (table.south east);
\end{tblrtikzbelow}%
\begin{tblrtikzabove}
  \draw[red3, thick]
    (2-2.north west) -- (2-3.south east)
    (2-2.south west) -- (2-3.north east);
\end{tblrtikzabove}%
\begin{tblr}{hline{2-3},vline{2-4}}
  1-1 & 1-2 & 1-3 & 1-4 \\
  2-1 & 2-2 & 2-3 & 2-4 \\
  3-1 & 3-2 & 3-3 & 3-4
\end{tblr}
\end{demohigh}

The second example below demonstrates corner nodes:

\begin{demohigh}
\begin{tblrtikzabove}
  \draw[color=white,thick]
       (h1-|v1) -- (h1-|v2) -- (h2-|v2)
    -- (h2-|v3) -- (h3-|v3) -- (h3-|v4)
    -- (h4-|v4) -- (h4-|v5) -- (h2-|v5)
    -- (h2-|v6) -- (h1-|v6);
\end{tblrtikzabove}%
\begin{tblr}{hlines={wd=4pt},vlines={wd=3pt}}
  1-1 & 1-2 & 1-3 & 1-4 & 1-5 \\
  2-1 & 2-2 & 2-3 & 2-4 & 2-5 \\
  3-1 & 3-2 & 3-3 & 3-4 & 3-5
\end{tblr}
\end{demohigh}

By \PP{tikz} intersection syntax, \fakeverb{h<i>-|v<j>} is
the instersection point of \KK{horder{<i>}} and \KK{vborder{<j>}}.

\section{Library \texttt{varwidth}}

To build a nice table, \PP{tabularray} need to measure the widths of cells.
By default, it uses \CC{\hbox} to measure the sizes.
This causes an error if a cell contains some vertical material, such as lists or display maths.

With \PP{varwidth} library, \PP{tabularray} will load \PP{varwidth} package,
add a new inner specification \K{measure}, and set \KV{measure=vbox}
so that it will use \CC{\vbox} to measure cell widths.

\begin{demohigh}
\begin{tblr}{hlines,measure=vbox}
  Text Text Text Text Text Text Text
  \begin{itemize}
    \item List List List List List List
    \item List List List List List List List
  \end{itemize}
  Text Text Text Text Text Text Text \\
\end{tblr}
\end{demohigh}

From version 2022A (2022-03-01), you can remove extra space above and below lists,
by adding option \KV{stretch=-1}.
The following example also needs \PP{enumitem} package and its \K{nosep} option:

{\centering\begin{tblr}{
  hlines,vlines,rowspec={Q[l,t]Q[l,b]},
  measure=vbox,stretch=-1,
}
  \begin{itemize}[nosep]
    \item List List List List List
    \item List List List List List List
  \end{itemize} & oooo \\
  \begin{itemize}[nosep]
    \item List List List List List
    \item List List List List List List
  \end{itemize} & gggg \\
\end{tblr}\par}

%% BUG: there is extra vertical space at the beginning of the first cell if I use demohigh
%\begin{demohigh}
\begin{codehigh}
\begin{tblr}{
  hlines,vlines,rowspec={Q[l,t]Q[l,b]},
  measure=vbox,stretch=-1,
}
  \begin{itemize}[nosep]
    \item List List List List List
    \item List List List List List List
  \end{itemize} & oooo \\
  \begin{itemize}[nosep]
    \item List List List List List
    \item List List List List List List
  \end{itemize} & gggg \\
\end{tblr}
\end{codehigh}
%\end{demohigh}

Note that option \KV{stretch=-1} also removes struts from cells, therefore it may not work well
in \PP{tabularray} environments with \KV{rowsep=0pt}, such as
\EE{booktabs}/\EE{longtabs}/\EE{talltabs} environments from \LL{booktabs} library.

From version 2025A, \KK{measure} key also accepts an experimental \VV{vstore} value.
With \KV{measure=vstore}, \PP{tabularray} also measures cells with \CC{\vbox},
but it will store the boxes for later use,
which is necessary to make \CC{\lTblrMeasuringBool} status correct.

From version 2025A, the setting of \KK{measure} key also applies to subtables.

\section{Library \texttt{zref}}

From version 2022D, you can load \LL{zref} library
to make \CC{\zref} and \EE{longtblr} work together.

\end{document}
