% -*- coding: utf-8 -*-
% !TEX program = lualatex
\documentclass[oneside]{book}

% -*- coding: utf-8 -*-
% !TEX program = lualatex

\newcommand*{\myversion}{2024A}
\newcommand*{\mylpad}[1]{\ifnum#1<10 0\the#1\else\the#1\fi}

\usepackage[a4paper,margin=2.5cm]{geometry}

\setlength{\parindent}{0pt}
\setlength{\parskip}{4pt plus 1pt minus 1pt}

\ExplSyntaxOn
\sys_ensure_backend:
\debug_on:n { check-declarations }
\usepackage{tabularray}
\debug_off:n { check-declarations }
\ExplSyntaxOff

\usepackage{codehigh} % https://ctan.org/pkg/codehigh
\usepackage{array,multirow,amsmath}
\usepackage{chemmacros,environ}
\usepackage{enumitem}

\usepackage[firstpage=true]{background}
\backgroundsetup{contents={}}

\UseTblrLibrary{
  amsmath,booktabs,counter,diagbox,functional,siunitx,varwidth
}

\usepackage{hyperref}
\hypersetup{
  colorlinks=true,
  urlcolor=blue3,
  linkcolor=blue3,
}

\usepackage{tcolorbox}
\tcbset{sharp corners, boxrule=0.5pt, colback=red9}

\usepackage{float}
%\usepackage{enumerate}

\setcounter{tocdepth}{1}

\newcommand*{\K}[1]{\texttt{#1}}
\newcommand*{\V}[1]{\texttt{#1}}
\newcommand*{\None}{$\times$}

\NewTblrEnviron{newtblr}
\SetTblrOuter[newtblr]{long}
\SetTblrInner[newtblr]{
  hlines = {white}, column{1,2} = {co=1}, colsep = 5pt,
  row{odd} = {brown8}, row{even} = {gray8},
  row{1} = {fg=white, bg=purple2, font=\bfseries\sffamily},
}

\NewTblrEnviron{spectblr}
\SetTblrOuter[spectblr]{long}
\SetTblrInner[spectblr]{
  hlines = {white}, column{2} = {co=1}, colsep = 5pt,
  row{odd} = {brown8}, row{even} = {gray8},
  row{1} = {fg=white, bg=purple2, font=\bfseries\sffamily},
  rowhead = 1,
}

\newcommand{\mywarning}[1]{%
  \begin{tcolorbox}
  The interfaces in this #1 should be seen as
  \textcolor{red3}{\bfseries experimental}
  and are likely to change in future releases, if necessary.
  Don’t use them in important documents.
  \end{tcolorbox}
}

%\renewcommand*{\thefootnote}{*}

\colorlet{highback}{azure9}
\CodeHigh{language=latex/table,style/main=highback,style/code=highback}
\NewCodeHighEnv{code}{style/main=gray9,style/code=gray9}
\NewCodeHighEnv{demo}{style/main=gray9,style/code=gray9,demo}

%\CodeHigh{lite}

\CodeHigh{lite}
\setcounter{chapter}{2}

\begin{document}

\chapter{Extra Interfaces}
\label{chap:extra}

In general, \verb!tblr! environment can accepts both inner and outer specifications:

\begin{codehigh}
\begin{tblr}[<outer specs>]{<inner specs>}
  <table body>
\end{tblr}
\end{codehigh}

\textbf{Inner specifications} are all specifications written in the \underline{mandatory} argument
of \verb!tblr! environment, which include new interfaces described in Chapter \ref{chap:basic}.

\textbf{Outer specifications} are all specifications written in the \underline{optional} argument
of \verb!tblr! environment, most of which are used for long tables (see Chapter \ref{chap:long}).

You can use \verb!\SetTblrInner! and \verb!\SetTblrOuter! commands
to set default inner and outer specifications of tables, respectively (see Section \ref{sec:default}).

\section{Inner Specifications}

In addition to new interfaces in Chapter \ref{chap:basic},
there are several inner specifications which are described in Table~\ref{key:inner}.

\begin{spectblr}[
  caption = {Keys for Inner Specifications},
  label = {key:inner},
]{}
  Key & Description and Values & Initial Value \\
  \K{rulesep} & space between two hlines or vlines & \V{2pt} \\
  \K{stretch} & stretch ratio for struts added to cell text & \V{1} \\
  \K{abovesep} & set vertical space above every row & \V{2pt} \\
  \K{belowsep} & set vertical space below every row & \V{2pt} \\
  \K{rowsep} & set vertical space above and below every row & \V{2pt} \\
  \K{leftsep} & set horizontal space to the left of every column & \V{6pt} \\
  \K{rightsep} & set horizontal space to the right of every column & \V{6pt} \\
  \K{colsep} & set horizontal space to both sides of every column & \V{6pt} \\
  \K{hspan} & horizontal span algorithm: \V{default}, \V{even}, or \V{minimal} & \V{default} \\
  \K{vspan} & vertical span algorithm: \V{default} or \V{even} & \V{default} \\
  %\K{verb} & you need this key to use verb commands & \None \\
  \K{baseline} & set the baseline of the table & \V{m} \\
\end{spectblr}

\subsection{Space between Double Rules}

The following example shows that we can replace \verb!\doublerulesep! parameter with \verb!rulesep! key.
\nopagebreak
\begin{demohigh}
\begin{tblr}{
 colspec={||llll||},rowspec={|QQQ|},rulesep=4pt,
}
 Alpha   & Beta  & Gamma  & Delta \\
 Epsilon & Zeta  & Eta    & Theta \\
 Iota    & Kappa & Lambda & Mu    \\
\end{tblr}
\end{demohigh}

\subsection{Minimal Strut for Cell Text}

The following example shows that we can replace \verb!\arraystretch! parameter with \verb!stretch! key.

\begin{demohigh}
\begin{tblr}{hlines,stretch=1.5}
 Alpha   & Beta  & Gamma  & Delta \\
 Epsilon & Zeta  & Eta    & Theta \\
 Iota    & Kappa & Lambda & Mu    \\
\end{tblr}
\end{demohigh}

By replacing stretch with row heights, we can get perfect vertical centering for your numerical tables.

\begin{demohigh}
\begin{tblr}{hlines, stretch=0, rows={ht=\baselineskip}}
  2021 & 2022 & 2023 \\
  0.4  & 0.5  & 0.6  \\
  1.1  & 2.2  & 3.3  \\
\end{tblr}
\end{demohigh}

\subsection{Rowseps and Colseps for All}

The following example uses \verb!rowsep! and \verb!colsep! keys to set padding for all rows and columns.
\nopagebreak
\begin{demohigh}
\SetTblrInner{rowsep=2pt,colsep=2pt}
\begin{tblr}{hlines,vlines}
 Alpha   & Beta  & Gamma  & Delta \\
 Epsilon & Zeta  & Eta    & Theta \\
 Iota    & Kappa & Lambda & Mu    \\
\end{tblr}
\end{demohigh}

\subsection{Hspan and Vspan Algorithms}

With \verb!hspan=default! or \verb!hspan=even!,
\verb!tabularray! package will compute column widths from span widths.
But with \verb!hspan=minimal!, it will compute span widths from column widths.
The following examples show the results from different \verb!hspan! values.

\begin{demohigh}
\SetTblrInner{hlines, vlines, hspan=default}
\begin{tblr}{cell{2}{1}={c=2}{l},cell{3}{1}={c=3}{l},cell{4}{2}={c=2}{l}}
 111 111 & 222 222 & 333 333 \\
 12 Multi Columns Multi Columns 12 & & 333 \\
 13 Multi Columns Multi Columns Multi Columns 13 & & \\
 111 & 23 Multi Columns Multi Columns 23 & \\
\end{tblr}
\end{demohigh}

\begin{demohigh}
\SetTblrInner{hlines, vlines, hspan=even}
\begin{tblr}{cell{2}{1}={c=2}{l},cell{3}{1}={c=3}{l},cell{4}{2}={c=2}{l}}
 111 111 & 222 222 & 333 333 \\
 12 Multi Columns Multi Columns 12 & & 333 \\
 13 Multi Columns Multi Columns Multi Columns 13 & & \\
 111 & 23 Multi Columns Multi Columns 23 & \\
\end{tblr}
\end{demohigh}

\begin{demohigh}
\SetTblrInner{hlines, vlines, hspan=minimal}
\begin{tblr}{cell{2}{1}={c=2}{l},cell{3}{1}={c=3}{l},cell{4}{2}={c=2}{l}}
 111 111 & 222 222 & 333 333 \\
 12 Multi Columns Multi Columns 12 & & 333 \\
 13 Multi Columns Multi Columns Multi Columns 13 & & \\
 111 & 23 Multi Columns Multi Columns 23 & \\
\end{tblr}
\end{demohigh}

The following examples show the results from different \verb!vspan! values.
\nopagebreak
\begin{demohigh}
\SetTblrInner{hlines, vlines, vspan=default}
\begin{tblr}{column{2}={3.25cm}, cell{2}{2}={r=3}{l}}
  Column1 & Column2 \\
  Row1 & Long text that needs multiple lines.
         Long text that needs multiple lines.
         Long text that needs multiple lines. \\
  Row2 & \\
  Row3 & \\
  Row4 & Short text \\
\end{tblr}
\end{demohigh}

\begin{demohigh}
\SetTblrInner{hlines, vlines, vspan=even}
\begin{tblr}{column{2}={3.25cm}, cell{2}{2}={r=3}{l}}
  Column1 & Column2 \\
  Row1 & Long text that needs multiple lines.
         Long text that needs multiple lines.
         Long text that needs multiple lines. \\
  Row2 & \\
  Row3 & \\
  Row4 & Short text \\
\end{tblr}
\end{demohigh}

\subsection{Use Verbatim Commands}

%With \verb!verb! key, you can write \verb!\verb! commands in the cell text:
%
%\begin{demohigh}
%\begin{tblr}{hlines,verb}
%  20 & 30 & \verb!\hello{world}!40 \\
%  50 & \verb!\hello!60 & 70 \\
%\end{tblr}
%\end{demohigh}

The inner key \verb!verb! is obsolete from version 2023A, and will be removed in the future.
Instead you can use more reliable \verb!\fakeverb! command (see Section \ref{sec:fakeverb}).

\subsection{Set Baseline for the Table}

With \verb!baseline! key, you can set baseline for the table.
All possible values for \verb!baseline! are as follows:

\begin{center}
\begin{tblr}{width=0.6\textwidth,colspec={cX[l]},hlines}
  \V{t}    & align the table at the top \\
  \V{T}    & align the table at the first row \\
  \V{m}    & align the table at the middle, initial value \\
  \V{b}    & align the table at the bottom \\
  \V{B}    & align the table at the last row \\
  \V{<n>}  & align the table at row \V{<n>} (a positive integer) \\
\end{tblr}
\end{center}

If there is no hline above the first row, you get the same result with either \V{t} or \V{T}.
But you get different results if there are one or more hlines above the row:

\begin{demohigh}
Baseline\begin{tblr}{hlines,baseline=t}
 Alpha   & Beta  & Gamma  \\
 Epsilon & Zeta  & Eta    \\
 Iota    & Kappa & Lambda \\
\end{tblr}Baseline
\end{demohigh}

\begin{demohigh}
Baseline\begin{tblr}{hlines,baseline=T}
 Alpha   & Beta  & Gamma  \\
 Epsilon & Zeta  & Eta    \\
 Iota    & Kappa & Lambda \\
\end{tblr}Baseline
\end{demohigh}

The differences between \verb!b! and \verb!B! are similar to \verb!t! and \verb!T!.
In fact, these two values \verb!T! and \verb!B! are better replacements
for currently obsolete \verb!\firsthline! and \verb!\lasthline! commands.

\section{Outer Specifications}

Except for specifications to be introduced in Chapter \ref{chap:long},
there are several other outer specifications which are described in Table~\ref{key:outer}.

\begin{spectblr}[
  caption = {Keys for Outer Specifications},
  label = {key:outer},
]{}
  Key & Description and Values & Initial Value \\
  \K{baseline} & set the baseline of the table & \V{m} \\
  \K{long} & change the table to a long table & \None \\
  \K{tall} & change the table to a tall table & \None \\
  \K{expand} & you need this key to use verb commands & \None \\
  \K{expand+} & like \K{expand} but appends to previous values & \None \\
\end{spectblr}

\subsection{Set Baseline in Another Way}

You may notice that you can write \K{baseline} option as either an inner or an outer specification.
It is true that either way would do the job. But there is a small difference:
when \verb!baseline=t/T/m/b/B! is an outer specification,
you can omit the key name and write the value only.

\begin{demohigh}
Baseline\begin{tblr}[m]{hlines}
 Alpha   & Beta  & Gamma  \\
 Epsilon & Zeta  & Eta    \\
 Iota    & Kappa & Lambda \\
\end{tblr}Baseline
\end{demohigh}

\subsection{Long and Tall Tables}

You can change a table to long table by passing outer specification \K{long},
or change it to tall table by passing outer specification \K{tall} (see Chapter~\ref{chap:long}).
Therefore the following two tables are the same:
\begin{codehigh}
\begin{longtblr}{lcr}
  Alpha & Beta & Gamma
\end{longtblr}
\begin{tblr}[long]{lcr}
  Alpha & Beta & Gamma
\end{tblr}
\end{codehigh}

\subsection{Expand Macros First}

In contrast to traditional \verb!tabular! environment, \verb!tabularray! environments
need to see every \verb!&! and \verb!\\! when splitting the table body with \verb!l3regex!.
And you can not put cell text inside any table command defined with \verb!\NewTableCommand!.
But you could use outer key \verb!expand! to make \verb!tabularray! expand
every occurrence of any of the specified macros \underline{once} and \underline{in the given order} before splitting the table body.
Note that you \underline{can not} expand a command defined with \verb!\NewDocumentCommand!.
You can also use \verb!expand+! if you still want to keep the macros in the current \verb!expand! setting.

To expand a command without optional argument, you can define it with \verb!\newcommand!.

\begin{demohigh}
\newcommand*\tblrrowa{
  20 & 30 & 40 \\
}
\newcommand*\tblrrowb{
  50 & 60 & 70 \\
}
\newcommand*\tblrbody{
 \hline
  \tblrrowa
  \tblrrowb
 \hline
}
\SetTblrOuter{expand=\tblrbody\tblrrowa}
\begin{tblr}[expand+=\tblrrowb]{ccc}
 \hline
  AA & BB & CC \\
  \tblrbody
  DD & EE & FF \\
  \tblrbody
  GG & HH & II \\
 \hline
\end{tblr}
\end{demohigh}

To expand commands with optional arguments, you \underline{can not} define them
with \verb!\newcommand!. But you can define them with \verb!\NewExpandableDocumentCommand!,
and use option \verb!expand=\expanded! to do exhaustive expansions.

\begin{demohigh}
\NewExpandableDocumentCommand\yes{O{Yes}m}{\SetCell{bg=green9}#1}
\NewExpandableDocumentCommand\no{O{No}m}{\SetCell{bg=red9}#1}
\begin{tblr}[expand=\expanded]{hlines}
  What I get               & is below              \\
  \expanded{\yes{}}        & \expanded{\no{}}      \\
  \expanded{\yes[Great]{}} & \expanded{\no[Bad]{}}
\end{tblr}
\end{demohigh}

Note that you need to protect fragile commands (if any) inside them with \verb!\unexpanded! command.

\section{Default Specifications}%
\label{sec:default}

\verb!Tabularray! package provides \verb!\SetTblrInner! and \verb!\SetTblrOuter! commands
for you to change the default inner and outer specifications of tables.

In general different \verb!tabularray! environments (\verb!tblr!, \verb!talltblr!,
\verb!longtblr!, etc) could have different default specifications.
You can list the environments in the optional arguments of these two commands,
and they only apply to \verb!tblr! environment when the optional arguments are omitted.

In the following example, the first line draws all hlines and vlines for all \verb!tblr! tables
created afterwards, while the second line makes all \verb!tblr! tables created afterwards
vertically align at the last row.

\begin{codehigh}
\SetTblrInner{hlines,vlines}
\SetTblrOuter{baseline=B}
\end{codehigh}

And the following example sets zero \verb!rowsep! for all \verb!tblr! and \verb!longtblr! tables
created afterwards.

\begin{codehigh}
\SetTblrInner[tblr,longtblr]{rowsep=0pt}
\end{codehigh}

\section{New Tabularray Environments}

You can define new \verb!tabularray! environments using \verb!\NewTblrEnviron! command:

\begin{demohigh}
\NewTblrEnviron{mytblr}
\SetTblrInner[mytblr]{hlines,vlines}
\SetTblrOuter[mytblr]{baseline=B}
Text \begin{mytblr}{cccc}
 Alpha   & Beta  & Gamma  & Delta \\
 Epsilon & Zeta  & Eta    & Theta \\
 Iota    & Kappa & Lambda & Mu    \\
\end{mytblr} Text
\end{demohigh}

\section{New General Environments}

With \verb!+b! argument type of \verb!\NewDocumentEnvironment! command,
you can also define a new general environment based on \verb!tblr! environment
(note that there is an extra pair of curly braces at the end):

\NewDocumentEnvironment{fancytblr}{+b}{
  Before Text
  \begin{tblr}{hlines}
    #1
  \end{tblr}
  After Text
}{}
\begin{codehigh}
\NewDocumentEnvironment{fancytblr}{+b}{
  Before Text
  \begin{tblr}{hlines}
    #1
  \end{tblr}
  After Text
}{}
\end{codehigh}
\begin{demohigh}
\begin{fancytblr}
  One   & Two   & Three \\
  Four  & Five  & Six   \\
  Seven & Eight & Nine  \\
\end{fancytblr}
\end{demohigh}

\section{New Table Commands}

All commands which change the specifications of tables \textcolor{red3}{must} be defined with \verb!\NewTableCommand!.
The following example demonstrates how to define a new table command:

\begin{demohigh}
\NewTableCommand\myhline{\hline[0.1em,red5]}
\begin{tblr}{llll}
\myhline
 Alpha   & Beta  & Gamma   & Delta \\
 Epsilon & Zeta  & Eta     & Theta \\
 Iota    & Kappa & Lambda  & Mu    \\
\myhline
\end{tblr}
\end{demohigh}

\section{Odd and Even Selectors}

From version 2022A, child selectors \verb!odd! and \verb!even! accept an optional argument,
in which you can specify the start index and the end index of the children.

\begin{demohigh}
\begin{tblr}{
  cell{odd}{1} = {red9},
  cell{odd[4]}{2} = {green9},
  cell{odd[3-X]}{3} = {blue9},
}
  Head   & Head    & Head   \\
  Talk A & Place A & Date A \\
  Talk B & Place B & Date B \\
  Talk C & Place C & Date C \\
  Talk D & Place D & Date D \\
  Talk E & Place E & Date E \\
  Talk F & Place F & Date F \\
  Talk G & Place G & Date G \\
  Talk H & Place H & Date H \\
\end{tblr}
\end{demohigh}

\begin{demohigh}
\begin{tblr}{
  cell{even}{1} = {yellow9},
  cell{even[4]}{2} = {cyan9},
  cell{even[3-X]}{3} = {purple9},
}
  Head   & Head    & Head   \\
  Talk A & Place A & Date A \\
  Talk B & Place B & Date B \\
  Talk C & Place C & Date C \\
  Talk D & Place D & Date D \\
  Talk E & Place E & Date E \\
  Talk F & Place F & Date F \\
  Talk G & Place G & Date G \\
  Talk H & Place H & Date H \\
\end{tblr}
\end{demohigh}

\section{Counters and Lengths}

Counters \verb!rownum!, \verb!colnum!, \verb!rowcount!, \verb!colcount! can be used in cell text:
\nopagebreak
\begin{demohigh}
\begin{tblr}{hlines}
 Cell[\arabic{rownum}][\arabic{colnum}] & Cell[\arabic{rownum}][\arabic{colnum}] &
 Cell[\arabic{rownum}][\arabic{colnum}] & Cell[\arabic{rownum}][\arabic{colnum}] \\
 Row=\arabic{rowcount}, Col=\arabic{colcount} &
 Row=\arabic{rowcount}, Col=\arabic{colcount} &
 Row=\arabic{rowcount}, Col=\arabic{colcount} &
 Row=\arabic{rowcount}, Col=\arabic{colcount} \\
 Cell[\arabic{rownum}][\arabic{colnum}] & Cell[\arabic{rownum}][\arabic{colnum}] &
 Cell[\arabic{rownum}][\arabic{colnum}] & Cell[\arabic{rownum}][\arabic{colnum}] \\
\end{tblr}
\end{demohigh}

Also, lengths \verb!\leftsep!, \verb!\rightsep!, \verb!\abovesep!, \verb!\belowsep! can be used in cell text.

\section{Tracing Tabularray}

To trace internal data behind \verb!tblr! environment, you can use \verb!\SetTblrTracing! command.
For example, \verb!\SetTblrTracing{all}! will turn on all tracings,
and \verb!\SetTblrTracing{none}! will turn off all tracings.
\verb!\SetTblrTracing{+row,+column}! will only tracing row and column data.
All tracing messages will be written to the log files.

\end{document}
