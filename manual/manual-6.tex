% -*- coding: utf-8 -*-
% !TEX program = lualatex
\documentclass[oneside]{book}

% -*- coding: utf-8 -*-
% !TEX program = lualatex

\newcommand*{\myversion}{2024A}
\newcommand*{\mylpad}[1]{\ifnum#1<10 0\the#1\else\the#1\fi}

\usepackage[a4paper,margin=2.5cm]{geometry}

\setlength{\parindent}{0pt}
\setlength{\parskip}{4pt plus 1pt minus 1pt}

\ExplSyntaxOn
\sys_ensure_backend:
\debug_on:n { check-declarations }
\usepackage{tabularray}
\debug_off:n { check-declarations }
\ExplSyntaxOff

\usepackage{codehigh} % https://ctan.org/pkg/codehigh
\usepackage{array,multirow,amsmath}
\usepackage{chemmacros,environ}
\usepackage{enumitem}

\usepackage[firstpage=true]{background}
\backgroundsetup{contents={}}

\UseTblrLibrary{
  amsmath,booktabs,counter,diagbox,functional,siunitx,varwidth
}

\usepackage{hyperref}
\hypersetup{
  colorlinks=true,
  urlcolor=blue3,
  linkcolor=blue3,
}

\usepackage{tcolorbox}
\tcbset{sharp corners, boxrule=0.5pt, colback=red9}

\usepackage{float}
%\usepackage{enumerate}

\setcounter{tocdepth}{1}

\newcommand*{\K}[1]{\texttt{#1}}
\newcommand*{\V}[1]{\texttt{#1}}
\newcommand*{\None}{$\times$}

\NewTblrEnviron{newtblr}
\SetTblrOuter[newtblr]{long}
\SetTblrInner[newtblr]{
  hlines = {white}, column{1,2} = {co=1}, colsep = 5pt,
  row{odd} = {brown8}, row{even} = {gray8},
  row{1} = {fg=white, bg=purple2, font=\bfseries\sffamily},
}

\NewTblrEnviron{spectblr}
\SetTblrOuter[spectblr]{long}
\SetTblrInner[spectblr]{
  hlines = {white}, column{2} = {co=1}, colsep = 5pt,
  row{odd} = {brown8}, row{even} = {gray8},
  row{1} = {fg=white, bg=purple2, font=\bfseries\sffamily},
  rowhead = 1,
}

\newcommand{\mywarning}[1]{%
  \begin{tcolorbox}
  The interfaces in this #1 should be seen as
  \textcolor{red3}{\bfseries experimental}
  and are likely to change in future releases, if necessary.
  Don’t use them in important documents.
  \end{tcolorbox}
}

%\renewcommand*{\thefootnote}{*}

\colorlet{highback}{azure9}
\CodeHigh{language=latex/table,style/main=highback,style/code=highback}
\NewCodeHighEnv{code}{style/main=gray9,style/code=gray9}
\NewCodeHighEnv{demo}{style/main=gray9,style/code=gray9,demo}

%\CodeHigh{lite}

\CodeHigh{lite}
\setcounter{chapter}{5}

\begin{document}

\chapter{Tips and Tricks}

\section{Control Horizontal Alignment}

You can control horizontal alignment of cells in \texttt{tabularray} with
\href{https://www.ctan.org/pkg/ragged2e}{\texttt{ragged2e}} package,
by redefining some of the following commands:

\begin{codehigh}
\RenewDocumentCommand\TblrAlignBoth{}{\justifying}
\RenewDocumentCommand\TblrAlignLeft{}{\RaggedRight}
\RenewDocumentCommand\TblrAlignCenter{}{\Centering}
\RenewDocumentCommand\TblrAlignRight{}{\RaggedLeft}
\end{codehigh}

Please read the documentation of \texttt{ragged2e} package for more details of
their alignment commands.

\section{Use Safe Verbatim Commands}%
\label{sec:fakeverb}

Due to the limitations of TeX,
we are not able to make \fakeverb{\verb} command behave well inside \texttt{tabularray} tables.
As a replacement, you may use \fakeverb{\fakeverb} command from \href{https://www.ctan.org/pkg/codehigh}{\texttt{codehigh}} package.

The \verb|\fakeverb| command will remove the backslashes in the following control symbols before
typesetting its content: \fakeverb{\\\\}, \fakeverb{\\\{}, \fakeverb{\\\}}, \fakeverb{\\\#}, \fakeverb{\\\^} and \texttt{\textbackslash\textvisiblespace}, \fakeverb{\\\%}.
Also the argument of \verb|\fakeverb| command need to be enclosed with curly braces.
Therefore it could be safely used inside \verb|tabularray| tables and other LaTeX commands.

Here is an example of using \verb!\fakeverb! commands inside a \verb|tblr| environment:

\begin{demohigh}
\begin{tblr}{hlines}
  Special & \fakeverb{\abc{}$&^_^uvw 123} \\
  Spacing & \fakeverb{\bfseries\ \#\%}    \\
  Nesting & \fbox{\fakeverb{$\left\\\{A\right.$\#}}
\end{tblr}
\end{demohigh}

In the above example, balanced curly braces and control words (such as \verb!\bfseries!)
need not to be escaped---only several special characters need to be escaped.
Please read the documentation of \texttt{codehigh} package for more details of
\verb|\fakeverb| commands.%
\footnote{By the way, \fakeverb{\EscVerb} command from
\href{https://www.ctan.org/pkg/fvextra}{\texttt{fvextra}} package is similar to
\fakeverb{\fakeverb} command, but with \fakeverb{\EscVerb} you need to escape every control word.}

\end{document}
